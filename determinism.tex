\documentclass[a4paper, 12pt]{article}
\usepackage[T1]{fontenc}
\usepackage[utf8]{inputenc}
\usepackage{babel}
\usepackage[raggedright]{titlesec}
\usepackage{blindtext}
\titleformat{\paragraph}[hang]{\normalfont\normalsize\bfseries}{\theparagraph}{1em}{}
\titlespacing*{\paragraph}{0pt}{3.25ex plus 1ex minus .2ex}{0.5em}
\setcounter{section}{0}
\begin{document}
\section{Determinism}
\paragraph{Preface}
Most things (and arguably all things) in this universe are out of our control. We do not have free will to choose the place of our birth. We do not have free will to choose our favorite flavor of ice cream. We do not have free will to believe in particular conclusion. We do not have free will to live or to die. However, even though the vast majority of events in the world have material causes that can be examined by science, the belief of determinism seems to suggest that all events in the universe are caused by events prior in adherence to complex laws of the universe that may never be fully understood by human beings. Consequently, even if one event within our human experience can be shown to be hypothetically unable to be predicted, then it appears that human beings do have some free will to focus their efforts between given choices even though we as human beings may not be able to create more choices than those that are given to us. \\
\break
My goal to show that although the philosophy of determinism appears to be valid for most of human experience due to the latest breakthroughs in science and due to the truth that we as human beings have no control for most of our circumstances (such as an earthquake or our genetical strengths), the idea of determinism seems to become very unintuitive in ordinary situations where a simple choice is required such as the one that I describe next. \\
\paragraph{Situation}
Suppose that you are sitting down at a table and are told that you must grab either the red or blue ball within the next minute. For the sake of this situation, suppose that no other choices can be made because other situations can be devised where only binary choices are possible (e.g. either you stay in this room or you leave it). As long as it can be shown that there is no determined choice that you will make, then determinism (in the sense that every event in the future are caused by the laws of the universe and the present state of the world) appears to be false. \\
\pagebreak
\paragraph{Argument}
Premise 1: Determinism is true. \\
\break
Premise 2: If Determinism is true, there exists only one possible future state of the world. All other supposed future states of the world other than the determined future are not logically possible given the laws of the universe and the state of the world in the present moment. \\
\break
Premise 3: If there exists only one possible future state of the world, I can hypothetically have knowledge of events in a future state of the world such as whether I will choose the red or blue ball in the situation. \\
\break
Premise 4: If I can hypothetically have knowledge of a particular event in a future state of the world, I can hypothetically choose the ball that I was not determined to choose in the situation. \\
\break
Premise 5: If I can hypothetically choose the ball that I was not determined to choose, then I was not determined to choose a ball (ie. there is no determined future state of the world) \\
\break
Contradiction: Premise 5 contradicts Premise 2 because Premise 1+2 say that a determined future state of the world does exist. \\
\paragraph{Objection To Premise 2}
If a critic argues against premise 2, then it appear that the person does not understand determinism the way that I am using it in this argument. Any objection to premise 1 would come down to definitional disagreements. \\
\paragraph{Objection To Premise 3}
A critic may point out premise 3 as the fallacy in the argument and argue instead that any knowledge of a future state of the world is not logically possible. Note that it is not sufficient enough fora critic to show that knowledge of a future state of the world is not empirically possible because the argument depends on only hypothetical knowledge of a future state of the world. \\
\break
However, if the critic argues that knowledge of the future is not logically possible because it would allow a person to act against the determined future, it then appears that the critic is only saying this to maintain that determinism is true. In other words, it appears that the critic is only precluding the validity of premise 3 because its validity would mean that determinism was false. \\
\break
Furthermore, if one determined future state of the world does exist, my knowledge of it does not intuitively suggest that a logical contradiction would result. In fact, since a determined future of the world cannot change no matter what (since there are no agents acting outside of causality in a deterministic world), it seems that my knowledge of it could could in no way affect its validity as the determined future. \\
\paragraph{Objection To Premise 4}
Arguing against premise 4 would appear to be the most unintuitive response to the argument because doing so would imply that knowledge of a future event causes something about the world to coerce me to act in accordance with that future event. For example, if I hypothetically knew that I will choose the red ball, and began to move to pick up the blue ball, it would be unintuitive to suggest that something about nature would move my arm and open my hands to grasp the red ball. \\
\break
A person can informally test the validity of premise 3 by having someone else tell him/her ``I know that you will choose the red ball'' and choose to instead grasp the blue ball. In this case, the other person acts as the hypothetical source of knowledge of the future. \\
\break
Also, note that in this situation, it is not posssible for me to be predetermined to interfere with my knowledge of the future if I did indeed interfere since I cannot be both predetermined to choose a particular ball and predeteremined to interefere with my knowledge that I will choose a particular ball. If a critic suggests that if I choose to interfere with the determined ball then I was determined to interfere, then that contradicts my having true knowledge of the future in the first place. \\
\paragraph{Objection To Premise 5}
I do not anticipate any objection to premise 5 because if an event is predetermined, any other hypothetical possibility is logically impossible given the laws of the universe and the present state of the world. \\
\paragraph{Objection To Premise 1}
If premise 2+3+4+5 are all believed to be valid, then premise 1 must be false ie. determinism is false because there is no one determined state of the future that must follow from the laws of the universe and the present state of the world. \\
\paragraph{Note}
Advances in neuroscience show that before a choice is made such as moving one's left hand towards a red ball, a person's brain state reflects that choice just moments before the actual physical choice. Some people have used this study to show that our actual choices are caused by brain states. However, this study does not actually give evidence for determinism because a theory of mind-body duality can also explain this by suggesting that the mind of an individual as the power to generate a certain brain state and that the physical choice is delayed because of the inherent time it takes between a person's decision and expression of that decision in the form of choosing between the red or blue ball. \\
\paragraph{Summary}
I do not intend for this argument to be proof of free will or a proof of determinism. Nothing of the world can be proved with the certainty in mathematics because everything in the world is known a posteriori and cannot be known through a relation of ideas as Hume notes. I only put forth this argument as evidence that determinism is unintuitive not because we can rarely act as if we have no freedom to choose, but because it seems that if there was indeed a deterimined future, then we could have knowledge of it and therefore change the future. The common response to this argument seems to be to escape the argument by precluding the possibility of knowledge of the future due to our ability to interfere if we obtained such knowledge. However, the very fact that it seems we would have the ability to interfere suggests the non-existence of a deterimined future and opens up the possibility for non-compatibilist formulations of free will. \\
\end{document}
